\documentclass[a4paper, 12pt]{article}
\usepackage[english]{babel}
\usepackage[backend=biber, style=apa]{biblatex}
\addbibresource{csc_research.bib}
\usepackage[T1]{fontenc}
\usepackage[utf8]{inputenc}
\usepackage{dsfont}
\usepackage{authblk}
\usepackage{amstext}
\usepackage{amssymb}
\usepackage{amsmath}
\usepackage{mathptmx}
\usepackage{setspace}
\usepackage{appendix}
\usepackage{tabularx}
\usepackage{booktabs}
\usepackage{caption}
\captionsetup{singlelinecheck=off,font=small,labelfont=bf}
\usepackage[nolists,tablesfirst,nomarkers]{endfloat}
\newcommand{\RR}{\raggedright\arraybackslash}
\newcommand{\RL}{\raggedleft\arraybackslash}
\newcommand{\CC}{\centering\arraybackslash}
\setlength{\parindent}{2em}
\setlength{\parskip}{0em}


\title{Employing Machine-Learning Approaches in Predicting Incomes of Recent College Graduates}
\author[1]{Benjamin Skinner}
\author[2]{William Doyle}
\author[3]{Olivia Morales}
\affil[1, 3]{University of Florida}
\affil[2]{Vanderbilt University}

\date{\today}

\begin{document}

\maketitle

\doublespacing

\begin{center}
\section*{Abstract}
\end{center}

Using a principled machine-learning approach, we predict recent
college graduates' earnings using data from the College
Scorecard. Early results support the predictive capabilities of
institutional characteristics like school classification and overall
debt repayment rates on recent graduate earnings.

\vspace{5mm}

keywords: \emph{machine learning, postsecondary earnings, return on investment}


\section*{Introduction}

The broad objective of this project involves the prediction of program
earnings for recent college graduates using common
institutional/program variables available via the College Scorecard
website. Econometric approaches to predicting earnings after
graduation are not uncommon in the higher education literature, as
many researchers in the field have attempted to support college-going
behavior due to its generous return on
investment. \cite{Oreopoulous_Petronijevic_2013} take a comprehensive look
at the research available on market returns to higher education,
reviewing 30 years of literature that ultimately demonstrate an
economic advantage and higher earnings potential for those individuals
with a college education. However, \cite{Carnevale_etal_2011} notes an
important caveat for this general earnings boost: the potential
earnings increase depends on the type of degree/credential earned,
program of study, etc.

The creation and publication of the College Scorecard by the
U.S. Department of Education presented an opportunity for families to
identify the institutions that provided the best labor outcomes for
their students with the least amount of financial burden
\cite{obama_2013}. Made publicly available in 2015, the data in the
College Scorecard (while illuminating varied institutional
characteristics), did not generally produce the kind of impact the
Obama administration envisioned and went mostly underutilized. It also
fell short of providing complete data profiles of
institutional/program characteristics, as much of the data eventually
published were missing or privacy suppressed.

Despite its shortcomings, the College Scorecard data have been used in
conjunction with common econometric approaches to evaluate student
responsiveness to the Scorecard. In particular, \cite{hurwitz_student_2018}
utilizes a DID model that demonstrates the decision-making changes in
generally well-resourced high school students after the publication of
the Scorecard, directing their SAT scores to schools that, on average,
had higher median earnings for graduates; the two other hallmarks of
the Scorecard (graduation rates and average costs), produced virtually
no change in SAT score-sending behaviors. Other higher
education/economics researchers have adopted econometric
methodological approaches while engaging the earnings data available
on the Scorecard in particular institutional and program contexts
\cite{boland_effect_2021, elu_earnings_2019, mabel_value_2020,
seaman_assessing_2017}. It's important, however, to highlight the
tendency of econometric methods to misspecify models and lend itself to
selector/researcher bias \cite{Imbens_2004}.

Machine learning, in contrast, allows the computer and corresponding
algorithms to determine the model and ultimately train the model to
promote increased accuracy. While commonly associated with convoluted
computational statistics and computer programming methods, it has
crept into the education (particularly higher education) field to
bolster model accuracy and potential estimates in quantitative higher
education studies. In particular, the last 6-7 years have seen an
uptick in higher education research projects utilizing machine
learning methods \cite{aulck2017predicting, savvas_etal_2021, Zeineddine_2021}. 
With this increase in prominence, how does this
project stand out? 

This project fills a dire gap in higher education literature by not
only utilizing the myriad institutional data points available on the
Scorecard, but marrying these data with novel machine learning
techniques that improve the predictive capacity of common
institutional characteristics in determining potential graduate
earnings.

\printbibliography
\end{document}

